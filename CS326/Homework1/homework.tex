\documentclass[11pt,a4paper,oneside]{article}

\usepackage{amsmath}
\usepackage{multirow}

\usepackage{caption}
\usepackage{color}
\usepackage{listings}
\usepackage{courier}
\lstset{
  basicstyle=\footnotesize\ttfamily, % Standardschrift
    %numbers=left,               % Ort der Zeilennummern
    numberstyle=\tiny,          % Stil der Zeilennummern
    %stepnumber=2,               % Abstand zwischen den Zeilennummern
    numbersep=5pt,              % Abstand der Nummern zum Text
    tabsize=2,                  % Groesse von Tabs
    extendedchars=true,         %
    breaklines=true,            % Zeilen werden Umgebrochen
    keywordstyle=\color{red},
    frame=b,         
    %        keywordstyle=[1]\textbf,    % Stil der Keywords
      %        keywordstyle=[2]\textbf,    %
      %        keywordstyle=[3]\textbf,    %
      %        keywordstyle=[4]\textbf,   \sqrt{\sqrt{}} %
      stringstyle=\color{white}\ttfamily, % Farbe der String
      showspaces=false,           % Leerzeichen anzeigen ?
      showtabs=false,             % Tabs anzeigen ?
      xleftmargin=17pt,
    framexleftmargin=17pt,
    framexrightmargin=5pt,
    framexbottommargin=4pt,
    %backgroundcolor=\color{lightgray},
    showstringspaces=false      % Leerzeichen in Strings anzeigen ?        
}
\lstloadlanguages{% Check Dokumentation for further languages ...
  %[Visual]Basic
    %Pascal
    %C
    C++
    %XML
    %HTML
    %Java
}
%\DeclareCaptionFont{blue}{\color{blue}} 

%\captionsetup[lstlisting]{singlelinecheck=false, labelfont={blue}, textfont={blue}}
\usepackage{caption}
\DeclareCaptionFont{white}{\color{white}}
\DeclareCaptionFormat{listing}{\colorbox[cmyk]{0.43, 0.35, 0.35,0.01}{\parbox{\textwidth}{\hspace{15pt}#1#2#3}}}
\captionsetup[lstlisting]{format=listing,labelfont=white,textfont=white, singlelinecheck=false, margin=0pt, font={bf,footnotesize}}
\begin{document}

%title page%%%%%%%%%%%%%%%%%%%%%%%%

\title{CS326 \\
       Homework \#1}
\date{September 9, 2010}

\author{Joshua Gleason}

\maketitle
\thispagestyle{empty}

\pagebreak

%%%%%%%%%%%%%%%%%%%%%%%%%%%%%%%%%%%

\begin{enumerate}
\item
\begin{enumerate}
\item \#included is not a recognized token, so it is detected by the scanner. 
  \lstinputlisting{oneA.cc}
\item The \texttt{cout} statement must be followed by a semicolon, this is a syntax error.  
  \lstinputlisting{oneB.cc}
\item The \texttt{main} function should return an \texttt{int}, but it is trying to return a \texttt{char*}.
  \lstinputlisting{oneC.cc}
\item The pointer is pointing to the null address, yet the program tries to de-reference it.  This causes
  a dynamic semantic error.
  \lstinputlisting{oneD.cc}
\end{enumerate}


\end{enumerate}

\end{document}

