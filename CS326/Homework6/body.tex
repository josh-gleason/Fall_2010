%example
%\lstinputlisting{code.c}

\begin{enumerate}
\item %1
No, because macros in C are textural replacements done {\bf once} before the code is compiled.
This means that something like the following...

\vspace{20pt}
{\tt \#define fact(a) (a <= 1 ? 1 : a*fact(a-1))}
\vspace{20pt}

  will not work because if {\tt fact(a)} is used somewhere in the code, it is just replaced with
  the ternary expression which includes {\tt fact(a-1)}.  Because the substitution is only done
  once, this ``new'' {\tt fact} will be undefined.

\item %2
\begin{enumerate}
  \item %a
    The local variable {\tt y} appears the retain its value because although the memory freed on
    the stack, that same memory is then re-allocated when the function is called again.  Because
    the stack is not modified between calls, the bit values remain in tact.
  \item %b
    If some variable makes use of the memory where {\tt y} was allocated, then when {\tt p()} is
    called again, the first value of {\tt y} is lost.  For example...

    \lstinputlisting{problem2.c}

    The output for this code is...

    \vspace{20pt}
    {\tt 0 1065353216}
    \vspace{20pt}

\end{enumerate}

\newpage

\item %3
{\tt
\begin{verbatim}
procedure swap(a,b)
begin
  temp := a
  a := b
  b := a
end

\end{verbatim}
}
\begin{enumerate}
  \item
    If the parameter passing is by value, then swap(a,b) swaps the temporary variables inside
    the swap function, but these values are discarded when the function returns.  So after this
    function executes, {\tt a} and {\tt b} have not changed.  In fact it is impossible to change
    {\tt a} and {\tt b} during a function call, making a swap function impossible.
  \item
    If the parameter passing is by name, then swap(a,b) will work for non-array values.  However
    suppose that swap(a,x[a]) is called.  In this case the actual names can be replaced inside
    the swap function in order to simulate evaluation.
{\tt

\begin{verbatim}
  temp := a
  a := x[a]
  x[a] := temp
\end{verbatim}

}

  because {\tt a} has changed, {\tt x[a]} has also changed, so now {\tt x[a]} could be out of
  bounds, or could be changing some other index.  Although special cases can be written to
  fix this particular example.  Infinitely many more examples like this one exist with nested
  subscripts.  Therefore it is impossible to write a swap function when parameters are passed
  by name.
\end{enumerate}

\item %4

\begin{enumerate}
\item {\tt 11}

  Because pass by value does not allow parameters to be changed, nothing happens to {\tt a[0]}
  and {\tt a[1]} after during the function call to {\tt p()}.  Therefore the original value that
  were assigned still hold.

\item {\tt 31}
  
  When {\tt p()} is called with a[0] as both parameters, a reference to that location is held by
  both {\tt x} and {\tt y}.  When {\tt x++} and {\tt y++} are executed, both increase the value
  of {\tt a[0]}.  {\tt a[1]} is not changed in this function.

\item
  
  When {\tt a[i]} is restored, is {\tt a[1]} now set to a value because i was increased?  In
  that case the output is {\tt 12} because {\tt x} and {\tt y} are both {\tt 2} so whichever one
  is returned doesn't matter because the value will be {\tt 2}.  The other ambiguity is whether
  {\tt a[0]} gets ``restored'' or {\tt a[1]}.  If {\tt a[0]} is restored then the output is
  {\tt 21}.

\item {\tt 22}

  When {\tt p()} is called here, {\tt x} and {\tt y} both reference the name {\tt a[i]}.  When
  {\tt x++} is called i

\end{enumerate}

\item %5

When using optional parameters, if the programmer specifies the values as literals, then the
function call will be the same.  However if the optional parameters are filled with variables,
an extra operation must be executed in order to obtain the value that the variable is holding.
In the latter case, not specifying the values for the optional parameters is a bit faster.

\end{enumerate}
