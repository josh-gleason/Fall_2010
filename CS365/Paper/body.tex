  Many people today can not imagine a world without computers.  Computers play a role in every  aspect of our lives.  Everything from the transportation, to entertainment, to information storage and many more can be linked to computing. But who is responsible for this?  Although many people have a claim in the development of computers, a select few are chosen as the most prominent figures in computer science.  These people are awarded the prestigious Turing Award.

    The Turing Award is given out once every year to “an individual selected for contributions of a technical nature made to the computing community.”  In the computer science community, receiving the Turing Award is comparable to winning the Nobel Prize.  Although there are many notable people who have received this award, this presentation will outline one in particular.  The 1972 Turing Award was given to Edsger Dijkstra for fundamental contributions in the area of programming languages.

      Dijkstra was born May 11, 1930 in Rotterdam, Netherlands where he studied theoretical physics at Leidan University.  Throughout his college career his interests quickly shifted towards the new field of computing and in particular computer programming.  While studying, he became aware of the fact that combining studies in theoretical physics and programming was becoming increasingly difficult, and that he would need to make a decision of which path to pursue. Dijkstra claims that the decision came after a life changing conversation with his boss and soon to be doctoral advisor Adriaan van Wijngaarden at the Mathematical Center in Amsterdam.  After van Wigngaarden's encouragement, Dijkstra became determined to advance programming and make it a respectable discipline.

        While Dijkstra was best known for his essays on programming, he was also the author of a number of very famous algorithms.  Of these algorithms, the best known is his solution to the shortest path between two points in a graph; also known as Dijkstra's algorithm.  A graph is a type of structure in which a number of vertices; or nodes, are connected by edges of varying weight.  Dijkstra developed, and proved that the path of least weight between two vertices could be determined using a relatively algorithm. This method became widely used in programming theory as it is the most efficient method of finding a path through a graph with all positive weights.

          In discrete math, many problems similar to shortest path and lowest cost are explored in depth.  Examples of these problems are using methods of proof to show without a doubt, that there does not exist a more efficient means to accomplish some task.  In the classroom this is usually done by attempting to find a solution that is more efficient and showing that logically it can not be done.  This is known as proof by contradiction and is a very powerful proving technique.

            Dijkstra had insight into programming problems well beyond his time. Many researchers in computer science were preoccupied with the hardware, claiming that development of software would somehow become irrelevant. They believed this because one of the main opinions of the time was that programming was simply “optimizing the efficiency of the computational process.” Dijkstra, on the other hand, believed that as hardware became more and more powerful and complex, the process' that drove them would become equally complex.  He believed that development of structured programming languages would be imperative for the future of computer science.

              In 1965 Dijkstra published a paper stating his distaste of the GOTO statement in computer programming.  The GOTO statement would allow a programmer to redirect to flow of a program to any point in the code they desired.  Dijkstra argued that this made programs increasingly difficult to maintain, and also that the readability of the code was significantly reduced.  Although many modern programmers share this belief, Dijkstra is credited with first pushing for the abolishment of GOTO statements in all higher level programming languages.  In 1968, Dijkstra composed a letter to the Editor of Communications of the Association for Computing Machinery (ACM).  The letter is often times thought to mark a turning point in programming theory and the beginning of structured programming.

                Structured programming completely removes GOTO statements from programming and instead replaces them with sub-structures such as loops.  This alleviates many of the problems with determining a programs purpose by not allowing explicit jumps in control flow.  Dijkstra was a advocate of structured programming showing time and again how unstructured programming can have undesired effects on both the programmer and the user.  By structuring programs, bugs were less likely and code could be maintained much more easily.

                  During his Turing Award acceptance speech, Dijkstra endorsed the ALGOL 60 programming language and worked was a member of the group that first implemented a compiler for the language.  The ALGOL language was one of few imperative languages of the time, stressing structured programming it was the precursor to many of the modern programming languages including Pascal and C.  ALGOL has also become the method in which computer algorithms have often been described.

