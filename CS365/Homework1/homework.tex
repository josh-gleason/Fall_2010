\documentclass[11pt,a4paper,oneside]{article}

\usepackage{amsmath}
\usepackage{multirow}

\begin{document}

%title page%%%%%%%%%%%%%%%%%%%%%%%%

\title{CS365 \\
       Homework \#1}
\date{August 30, 2010}

\author{Joshua Gleason}

\maketitle
\thispagestyle{empty}

\pagebreak

%%%%%%%%%%%%%%%%%%%%%%%%%%%%%%%%%%%

\section*{Page 16}

\begin{enumerate}
\setcounter{enumi}{7}
\item %8
\begin{enumerate}
\item If you have the flu, then you miss the final examination.
\item You do not have the flu if and only if you pass the course.
\item If you miss the final examination, then you will not pass the course.
\item You have the flu, or you miss the examination, or you pass the course.
\item If you have the flu then you will not pass the course, or if you miss the final examination then you will not pass the course.
\item You have the flu and you miss the final examination, or you do not miss the final examination and you pass the course.
\end{enumerate}

\setcounter{enumi}{9}
\item %10
\begin{enumerate}
\item \( r \wedge \neg q \)
\item \( p \wedge q \wedge r \)
\item \( r \rightarrow p \)
\item \( p \wedge \neg q \wedge r \)
\item \( r \leftrightarrow ( p \vee q ) \)
\end{enumerate}

\setcounter{enumi}{33}
\item %34
\hfill \\ %skip line
\begin{center}
\begin{tabular}{c | c | c | c | l}
  p & q & r & s & \( ((p\rightarrow q)\rightarrow r)\rightarrow s\) \\
  \hline
  0 & 0 & 0 & 0 & 1 \\
  0 & 0 & 0 & 1 & 1 \\
  0 & 0 & 1 & 0 & 0 \\
  0 & 0 & 1 & 1 & 1 \\
  0 & 1 & 0 & 0 & 1 \\
  0 & 1 & 0 & 1 & 1 \\
  0 & 1 & 1 & 0 & 0 \\
  0 & 1 & 1 & 1 & 1 \\
  1 & 0 & 0 & 0 & 0 \\
  1 & 0 & 0 & 1 & 1 \\
  1 & 0 & 1 & 0 & 0 \\
  1 & 0 & 1 & 1 & 1 \\
  1 & 1 & 0 & 0 & 1 \\
  1 & 1 & 0 & 1 & 1 \\
  1 & 1 & 1 & 0 & 0 \\
  1 & 1 & 1 & 1 & 1 \\

\end{tabular}
\end{center}

\setcounter{enumi}{47}
\item %48
\begin{enumerate}
\item \( r \wedge \neg p \)
\item \( (r \wedge p ) \rightarrow q \)
\item \( \neg r \rightarrow \neg q \)
\item \( ( \neg p \wedge r ) \rightarrow q \)
\end{enumerate}

\end{enumerate}

\section*{Page 28}
\begin{enumerate}

\setcounter{enumi}{13}
\item \hfill \\ %14
\begin{tabular}{r c l l}
  \( ( \neg p \wedge (p \rightarrow q)) \rightarrow \neg q \) & \( \equiv \) & \( \neg( \neg p \wedge (p \rightarrow q) ) \vee \neg q \) & (From Table 7) \\
  & \( \equiv \) & \( \neg( \neg p \wedge ( \neg p \vee q) ) \vee \neg q \) & (From Table 7) \\
  & \( \equiv \) & \( (\neg (\neg p) \vee \neg(p \vee q)) \vee \neg q \) & (De Morgan's law) \\
  & \( \equiv \) & \( (p \vee \neg(p \vee q)) \vee \neg q \) & (Double negation law) \\
  & \( \equiv \) & \( (p \vee (\neg p \wedge \neg q)) \vee \neg q \) & (De Morgan's law) \\
\end{tabular}

Proposition can not be reduced further, therefore this is \textbf{not} a \textit{tautology}, it is a \textit{contingency}.

\setcounter{enumi}{15}
\item The propositions are true only when p and q are both positive, or both negative.

\begin{center}
\begin{tabular}{ c | c | c | l }
  p & q & \( p \leftrightarrow q \) & \( (p \wedge q) \vee (\neg p \wedge q \neg) \) \\
  \hline
  0 & 0 & 1 & 1 \\
  0 & 1 & 0 & 0 \\
  1 & 0 & 0 & 0 \\
  1 & 1 & 1 & 1 \\

\end{tabular}
\end{center}

\setcounter{enumi}{39}
\item %40
\( p \wedge q \wedge \neg r \)

\end{enumerate}

\end{document}

